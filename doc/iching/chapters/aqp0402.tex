\newpage
\maketitle
\begin{center}
\Large \textbf{第402章 统计套利} \quad 
\end{center}
\begin{abstract}
在本章中,我们将根据A股历史数据,完成基于交易对的统计套利交易策略。
\end{abstract}
\section{统计套利}
在股市中,同行业的两支股票,由于所处的宏观经济环境和市场环境相同,其股价之间一般会维持一个恒定比例,当出现某些特殊情况时,某一支股票会出现比较大的波动,这时二者之间的股价比例将发生大的波动,但是会在较短的时间内,又回归到原来的比特。如果我们同时持有这两支股票S1和S2,当其中S1临时出现股价上涨高估时,我们就卖出S1的股票,买入S2的股票,当S1股价回落时,我们再将这次买入的S2股票卖掉,重新买入S1的股票,保持S1和S2股票的持有数量和比例不变,但是由于S1出现股价波动,我们实际赚到了这次波的利润。S1出现低估时,需要反向操作,原理相同。\newline
基于交易对的统计套利策略包括三大问题:
\begin{itemize}
\item 交易对选择;
\item 交易策略制定;
\item 交易中的仓位管理;
下面我们分别来讨论这三个问题。
\end{itemize}
\subsection{交易对选择}
基于交易对的统计套利方法分为形成期和交易期。形成期就是根据历史行情数据,找出适合交易的交易对。交易对选择方法如下所示:
\begin{itemize}
\item 行业内匹配法:选择同行业规模和业务模式相近的企业,例如我们在本章例子中所举的,选择银行业相关股票;
\item 产业链配对法:选择同一产业链上下游企业配对,例如手机厂商和摄像头厂商配对;
\item 财务管理配对法:从基本面分析入手,选择市盈率、负债率、产品种类相似的企业进行配对;
\end{itemize}
国内沪深两市共有2780支股票,如果我们拿任意两种股票进行配对,会有3862810种组合,计算量过大。在实际应用中,我们通常选择沪深500指数股、上证50成份股和创业板指数股作为股票池,从中选择交易对。另外,有一些股票,同时在不同交易所上市,例如中石油就同时在国内A股和香港H股上市,由于二者表示同一家公司的股票,可以构成交易对。下面我们来看交易对的选择标准。
\subsubsection{最小距离法}
我们的目标是选择历史价差稳定的交易对,我们首先需要对股票价格数据进行标准化,假设第$i$支股票的价格序列为:
\begin{equation}
P^{i}_{t} \quad t \in \{1, 2, 3, ..., T \}
\label{e0402-stock-price-series}
\end{equation}
我们定义第$i$支股票在第$t$天的单期收益率为:
\begin{equation}
r^{i}_{t} = \frac{P^{i}_{t} - P^{i}_{t-1}}{ P^{i}_{t-1} }, \quad t=1, 2, 3, ..., T
\label{e0402-stock-invest-return-t-th-day}
\end{equation}
我们定义第$i$支股票在第$t$天的标准价格为:
\begin{equation}
\hat{p}^{i}_{t} = \sum _{\tau = 1}^{t} (1 + r_{\tau}^{i})
\label{e0402-stock-standard-price-t-th-day}
\end{equation}
我们定义股票X、Y的平方标准化价格差SSD为:
\begin{equation}
SSD_{X,Y} = \sum _{t=1}^{T} (\hat{p}_{t}^{X} - \hat{p}_{t}^{Y})^{2}
\label{e0402-stock-square-standard-price-diff}
\end{equation}
下面我们以上证50成份股为例,找出其出潜在的交易对:
\lstset{language=PYTHON, caption={基于最小距离法的上证50成份股交易对选择}, label={c000003}}
\begin{lstlisting}
#
import numpy as np
import pandas as pd

class TpApp(object):
    def __init__(self):
        self.name = 'app.tp.TpApp'
        self.stock_pool = [
                    '600000','600010','600015','600016','600018',
                    '600028','600030','600036','600048','600050',
                    '600104','600109','600111','600150','600518',
                    '600519','600585','600637','600795','600837',
                    '600887','600893','600999','601006',
                    '601088','601166','601169','601186',
                    '601318','601328','601390',
                    '601398','601601','601628','601668',
                    '601766','601857',
                    '601988','601989','601998']
        self.tpc = {}

    def startup(self):
        print('交易对应用')
        sh = pd.read_csv('./data/sh50p.csv', index_col='Trddt')
        sh.index = pd.to_datetime(sh.index)
        form_start = '2014-01-01'
        form_end = '2015-01-01'
        sh_form = sh[form_start : form_end]
        tpc = {}
        sp_len = len(self.stock_pool)
        for i in range(sp_len):
            for j in range(i+1, sp_len):
                tpc['{0}-{1}'.format(self.stock_pool[i], self.stock_pool[j])] = self.trading_pair(sh_form, self.stock_pool[i], self.stock_pool[j])                    
        self.tpc = sorted(tpc.items(), key=lambda x: x[1])
        for itr in self.tpc:
            print('{0}: {1}'.format(itr[0], itr[1]))

    def trading_pair(self, sh_form, stock_x, stock_y):        
        # 中国银行股价
        PAf = sh_form[stock_x]
        # 取浦发银行股价
        PBf = sh_form[stock_y]
        # 求形成期长度
        pairf = pd.concat([PAf, PBf], axis=1)
        form_len = len(pairf)
        return self.calculate_SSD(PAf, PBf)

    def calculate_SSD(self, price_x, price_y):
        if price_x is None or price_y is None:
            print('缺少价格序列')
            return 
        r_x = (price_x - price_x.shift(1)) / price_x.shift(1) [1:]
        r_y = (price_y - price_y.shift(1)) / price_y.shift(1) [1:]
        #hat_p_x = (r_x + 1).cumsum()
        hat_p_x = (r_x + 1).cumprod()
        #hat_p_y = (r_y + 1).cumsum()
        hat_p_y = (r_y + 1).cumprod()
        SSD = np.sum( (hat_p_x - hat_p_y)**2 )
        return SSD
\end{lstlisting}
\subsubsection{协整模型}
协整模型是更常用的一种选择交易对的方法。我们假设股票$X$在第$t$日价格为:
\begin{equation}
P_{t}^{X}, \quad t=1, 2, 3, ..., T
\label{e0402-stock-X-price-t}
\end{equation}
定义股票$X$的对数价格序列:
\begin{equation}
\log (P_{t}^{X}), \quad t=1, 2, 3, ..., T
\label{e0402-stock-X-log-price-t}
\end{equation}
一般股票$X$的对数价格序列是非平稳时间序列,我们定义股票$X$的一阶差分序列:
\begin{equation}
\log (P_{t}^{X}) - \log (P_{t-1}^{X}) = \log \Big( \frac{ P_{t}^{X} }{ P_{t-1}^{X} } \Big), \quad t=1, 2, 3, ..., T
\label{e0402-stock-X-one-rank-diff-series}
\end{equation}
如果一阶差分序列是平稳时间序列,则称股票$X$的价格序列为一阶单整序列。\newline
股票$X$的在第$t$期单期简单收益率为:
\begin{equation}
r_{t}^{X} = \frac{ P_{t}^{X} - P_{t-1}^{X} }{P_{t-1}^{X}} = \frac{ P_{t}^{X} }{P_{t-1}^{X}} - 1
\label{e0402-stock-X-t-term-simple-return}
\end{equation}
我们根据式\ref{e0402-stock-X-one-rank-diff-series}和式\ref{e0402-stock-X-t-term-simple-return}可得:
\begin{equation}
\frac{ P_{t}^{X} - P_{t-1}^{X} }{P_{t-1}^{X}} = \frac{ P_{t}^{X} }{P_{t-1}^{X}} = \log(r_{t}^{X} + 1) \approx r_{t}^{X}
\label{e0402-stock-X-return-with-price-diff}
\end{equation}
要判断两支股票历史价格是否具有协整关系,首先要检查两支股票的对数价格序列是否为一阶单整序列,根据式\ref{e0402-stock-X-return-with-price-diff}也就是检查两支股票单期收益率$r_{t}^{i}$是否为平稳时间序列。\newline
假设股票$X$的对数价格序列$\log \left( P_{t}^{X} \right)$和股票$Y$的对数价格序列$\log \left( P_{t}^{Y} \right)$均为一阶单整序列,则可以用最小二乘法构造一阶回归方程:
\begin{equation}
\log \left( P_{t}^{Y} \right) = \alpha + \beta \log \left( P_{t}^{X} \right) + \epsilon _{t}
\label{e0402-one-rank-regression-equation}
\end{equation}
如果我们可以确定回归系数$\hat{\alpha}$和$\hat{\beta}$,则可以得到残差序列:
\begin{equation}
\hat{\epsilon} _{t} = \log \left( P_{t}^{Y} \right) - \hat{\alpha} - \hat{\beta} \log \left( P_{t}^{X} \right) 
\label{e0402-one-rank-regression-equation-residue}
\end{equation}
如果我们可以证明$\hat{\epsilon} _{t}$是时间平稳序列,就可以证明股票$X$和股票$Y$的对数价格序列具有一阶协整关系。\newline
证明交易对是否具有协整关系的程序如下所示:
\lstset{language=PYTHON, caption={基于协整模型的交易对选择}, label={c0402-cointegration-trading-pair-choose}}
\begin{lstlisting}
#
import numpy as np
import pandas as pd
from arch.unitroot import ADF
import statsmodels.api as sm

class CitpEngine(object):
    def __init__(self):
        self.name = 'apps.tp.CitpEngine'

    def exp(self):
        print('协整模型试验')
        stock_x = '601988'
        stock_y = '600000'
        sh = pd.read_csv('./data/sh50p.csv', index_col='Trddt')
        sh.index = pd.to_datetime(sh.index)
        form_start = '2014-01-01'
        form_end = '2015-01-01'
        sh_form = sh[form_start : form_end]      
        # 中国银行股价
        PAf = sh_form[stock_x]
        #r = PAf - PAf.shift(1)
        #adf1 = ADF(r[1:])
        #print(adf1.summary().as_text())
        rst = self.calculate_adf(PAf)
        if not rst:
            print('中国银行收益率不是时间平稳序列')
            return 
        else:
            print('中国银行收益率是时间平稳序列 ^_^')
        # 取浦发银行股价
        PBf = sh_form[stock_y]
        rst = self.calculate_adf(PBf)
        if not rst:
            print('浦发银行收益率不是时间平稳序列')
            return 
        else:
            print('浦发银行收益率是时间平稳序列 ^_^')
        log_p_x = np.log(PAf)
        log_p_y = np.log(PBf)
        model = sm.OLS(log_p_y, sm.add_constant(log_p_x))
        result = model.fit()
        print(result.summary())
        alpha = result.params[0]
        beta = result.params[1]
        epsilon = log_p_y - alpha - beta * log_p_x
        # 残差项均值为零,需要使用trend='nc'
        adf_epsilon = ADF(epsilon, trend='nc')
        epsilon_stable = True
        for val in adf_epsilon.critical_values.values():
            if adf_epsilon.stat >= val:
                epsilon_stable = False
        if not epsilon_stable:
            print('残差项为非平稳时间序列')
            return 
        print('可以使用协整模型来创建交易对')

    def calculate_adf(self, p_x):
        log_p_x = np.log(p_x)
        ret_x = log_p_x.diff()[1:]
        adf_ret_x = ADF(ret_x)
        for val in adf_ret_x.critical_values.values():
            if adf_ret_x.stat >= val:
                return False
        return True
\end{lstlisting}
\subsection{强化学习环境}
我们为了支持策略回测和与深度元强化学习框架统一,我们采用强化学习环境来做交易回测。我们首先定义强化学习环境,如下所示:

代码解读如下:
\begin{itemize}
\item 第9行:action为[3,3],第1维代表对股票X的0-买入、1-保持、2-卖出,第2维表示0-自有资金、1-5倍杠杆、2-10倍杠杆,例如a[0]=0,a[1]=3,则为买入股票X,使用10倍杠杆,此时需要借钱来买入股票,再例如a[0]=2;
\item 第10$\sim$12行:股价最低为0元,最高为10000元,shape为(1, 5)就是之前5天的收盘价,为浮点数;
\item 第行:;
\end{itemize}






\section{当前}



\subsection{交易策略}
\subsubsection{最小距离法}
我们在形成期找到了历史价格具有最小距离的交易对,如前例子中的中国银行和浦发银行,假设交易对为股票$X$和$Y$,其标准化价格差为:
\begin{equation}
\hat{p}_{t}^{X} - \hat{p}_{t}^{Y}
\label{e0402-trading-pair-standard-price-diff}
\end{equation}
我们可以求出其均值$\mu$和标准差$\sigma$,由于我们例子中的银行股比较稳定,我们定义一个较小的阈值$\lambda=1.2$,我们定义正常的股价标准价差区间为:$\mu \pm \lambda \sigma$,当股价标准价差超出该区间时,我们进行如下操作:
\begin{itemize}
\item $\hat{p}_{t}^{X} - \hat{p}_{t}^{Y} < \mu - \lambda \sigma $时:我们卖出股票$Y$,全部买入股票$X$。当股价标准价差回归正常区间时,我们卖出刚买入的股票$X$,买入股票$Y$;
\item $\hat{p}_{t}^{X} - \hat{p}_{t}^{Y} > \mu - \lambda \sigma $时:我们卖出股票$X$,全部买入股票$Y$。当股价标准价差回归正常区间时,我们卖出刚买入的股票$Y$,买入股票$X$;
\end{itemize}
\subsubsection{协整模型}

算法:
仓位position\_state有以下四种情况:
\begin{itemize}
\item 0:股票X和股票Y均为空仓,价格水平为3和-3时;
\item 1:按比例分别持有X和Y,价格水平为0时;
\item 2:仅持有X,价格水平为2时;
\item 3:仅持有Y,价格水平为-2时;
\end{itemize}


\subsection{XGBoost强化学习}
\subsubsection{数据集定义}
\subsubsection{强化学习环境RxgbEnv}
















